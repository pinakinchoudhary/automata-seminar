\documentclass{beamer}
\usetheme{metropolis}           % Use metropolis theme
\usepackage{graphicx}
\usepackage{tikz}
\usepackage{amsmath, amssymb, amsfonts, amsthm}
\usepackage{bm}			% For bold maths \bm{...}
\usepackage{ wasysym }    % for \smiley
\usepackage{mathdots}   % for different ellipses
\usepackage{multirow}   % for merging multiple rows
\usepackage{hyperref}   % for links
\usepackage[linguistics]{forest} % for drawing trees
\usepackage{fancyvrb}   % for code insertion
\usepackage{float}		% for forcing figure position \begin{figure}[H]
\usepackage{xcolor}		% for colors
\usepackage{tcolorbox}	% for colorful boxes
\usepackage{comment} 	% for multiline comments
\usepackage{enumitem}	% for changing labeling of a ordered list. Very powerful [label = {  (\alph*).  }] for (a). (b). etc.
\setbeamercolor{block body}{bg=mDarkTeal!30}
\setbeamercolor{block title}{bg=mDarkTeal,fg=black!2}

\title{Büchi Automata}
\date{\today}
\author{Pinakin, Vansh, Aayush, Bhanu}
\institute{Indian Institute of Science, Bangalore}


\begin{document}
  \maketitle
  \tableofcontents  
  
  \section{Motivation}
  \begin{frame}{Motivation}
    Content for Motivation
  \end{frame}
  \section{Regular Expressions and $\omega$-Regular Expressions}
  \begin{frame}{Regular Expressions over $\Sigma$}
    \begin{block}{Definition}
      A regular expression is a sequence of characters that define a search pattern.
      \[\alpha := \phi \mid A \mid \alpha_1 + \alpha_2 \mid \alpha_1\cdot\alpha_2 \mid \alpha^*\]
      Where $A$ is an alphabet and $\phi$ is the empty string.
      \end{block}
      Some semantics used throughout the presentation:
      \begin{itemize}
        \item $\alpha_1 + \alpha_2$ is the union of $\alpha_1$ and $\alpha_2$
        \item $\alpha_1\cdot\alpha_2$ is the concatenation of $\alpha_1$ and $\alpha_2$
        \item $\alpha^*$ is the Kleene closure of $\alpha$
        \item $\mathcal{L}(\phi) = \phi$, $\mathcal{L}(A) = \{A\}$ and $\mathcal{L}(\epsilon)= \{\epsilon\}$
      \end{itemize}
  \end{frame}
  \begin{frame}{$\omega$-Regular Expression}
  \begin{definition}
    regular expressions + $\alpha^w$:=
    \[\text{for} \hspace{1mm} \mathcal{L} \subseteq \Sigma^*:\]
    \[\mathcal{L}^{\omega} = \{w_1w_2w_3\cdots: w_i \in \mathcal{L}\hspace{1mm} \forall i \geq 1\} \]
  \end{definition}
  Note: \[\mathcal{L}(\omega) \subseteq \Sigma^{\omega} \text{if} \epsilon \notin \mathcal{L}\]
  Kleene Star: 'finite repition'\\
  $\omega$-operator: 'infinite repition'
  \end{frame}
  \begin{frame}{Syntax and semantics of $\omega$-regular expressions}
    \begin{definition}
      syntax of $\omega$-regular expressions over alphabet $\Sigma$:
      \[\gamma  =  \alpha_1\cdot\beta_1^{\omega}+ \cdots + \alpha_n\cdot\beta_n^{\omega} \]
      $\alpha_i, \beta_i$ are regular expressions over $\Sigma$ such that $\epsilon \notin \mathcal{L}(\beta_i)$
    \end{definition}
    the language generated by $\gamma$ is: 
    \[\mathcal{L}_{\omega}(\gamma) = \bigcup_{1 \leq i \leq n}\mathcal{L}(\alpha_i)\mathcal{L}(\beta_i)^{\omega} \subseteq \Sigma^{\omega}\]
    \begin{example}
    the language of $(A^*B)^{\omega} = $ set of all infinite words containing  infinitely many $B$'s
    \end{example}
  \end{frame}
  \begin{frame}{$\omega$ Regular Language}
    \begin{block}{$\omega$ Regular Language}
      A language $\mathcal{L} \subseteq \Sigma^{\omega}$ is called regular iff there exists an $\omega$ regular expression
      such that $\mathcal{L} = \mathcal{L}_{\omega}(\gamma)$
    \end{block}
  \end{frame}
  \begin{frame}{Provide an $\omega$ regular expression for.....}
    alphabet $\Sigma = \{a, b\}$ such that the set of all infinite words contain finitel
     many $a$'s.
     \[(a + b)^*b^{\omega}\]
  \end{frame}
\end{document}